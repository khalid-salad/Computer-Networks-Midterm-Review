\section{Security -- Encryption, Integrity, Authentication, Certificate, HTTPS, and Pharming}
\subsection{Confidentality through Cryptography}
\begin{itemize}[nosep]
    \item Cryptography \emph{communication over insecure channel in the presence of adversaries}
    \item Central goal: how to encode information so that an adversary can't extract it, \dots but a friend can
    \item General premise: a \emph{key} is required for decoding
          \begin{itemize}[nosep]
              \item Give it to friends, keep it away from attackers
          \end{itemize}
    \item Two different categories of encryption
          \begin{itemize}[nosep]
              \item Symmetric: efficient, requires key distribution
              \item Asymmetric (Public Key): computationally expensive, but no key distribution problem
          \end{itemize}
\end{itemize}

\subsection{Symmetric Key Encryption}
\begin{itemize}[nosep]
    \item Same key for encryption and decryption
          \begin{itemize}[nosep]
              \item Both sender and receiver know key
              \item But adversary does not know key
          \end{itemize}
    \item For communication, problem is \emph{key distribution}
          \begin{itemize}[nosep]
              \item How do the parties (securely) agree on the key?
          \end{itemize}
    \item What can you do with a huge key? One-time pad
          \begin{itemize}[nosep]
              \item Huge key of random bits
          \end{itemize}
    \item To encrypt/decrypt, just \texttt{XOR} with the key!
          \begin{itemize}[nosep]
              \item \emph{Provably secure!} \dots provided:
                    \begin{itemize}[nosep]
                        \item You never reuse the key, \dots and it really is random/unpredictable
                    \end{itemize}
              \item Spies actually use these
          \end{itemize}
\end{itemize}

\subsection{Using Symmetric Keys}
\begin{itemize}[nosep]
    \item Both the sender and the receiver use the same symmetric key
\end{itemize}
\begin{figure}[H]
    \tikzsetnextfilename{symmetric-key}
    \begin{tikzpicture}
        \node (plaintext1) {Plaintext};
        \node[below=of plaintext1,draw,fill=yellow!40,ellipse,align=left] (enc) {Encrypt with\\\emph{secret} key};
        \node[right=6cm of enc,draw,fill=yellow!40,ellipse,align=left] (dec) {Decrypt with\\\emph{secret} key};
        \node[above=of dec] (plaintext2) {Plaintext};
        \path(enc) -- (dec) node[pos=0.5,minimum size=3cm,cloud,draw,fill=blue!40!white,yshift=-1cm] (internet) {};
        \node at ([yshift=1cm]internet.center) {Internet};
        \draw[->] (plaintext1) -- (enc);
        \draw (enc.south) -- ([yshift=-1cm]enc.south);
        \draw[->] ([yshift=-1cm]enc.south) -| (dec) node[pos=0.25,above] {Ciphertext};
        \draw[->] (dec) -- (plaintext2);
    \end{tikzpicture}
\end{figure}

\subsection{Asymmetric Encryption (Public Key)}
\begin{itemize}[nosep]
    \item Idea: use two \emph{different} keys, one to encrypt (\textbf{e}) and one to decrypt (\textbf{d})
          \begin{itemize}[nosep]
              \item A \emph{key pair}
          \end{itemize}
    \item Crucial property: knowing \textbf{e} does not give away \textbf{d}
    \item Therefore \textbf{e} can be public: everyone knows it!
    \item If Alice wants to send to Bob, she fetche's Bob's public key (say, from Bob's home page) and encrypts with it
          \begin{itemize}[nosep]
              \item \emph{Alice} can't decrypt what she is sending to Bob\dots
                    \begin{itemize}[nosep]
                        \item \dots but then, \emph{neither can anyone else} (except Bob)
                    \end{itemize}
          \end{itemize}
\end{itemize}

\subsection{Public Key/Asymmetric Encryption}
\begin{itemize}[nosep]
    \item Sender uses receiver's \emph{public} key
          \begin{itemize}[nosep]
              \item Adverised to everyone
          \end{itemize}
    \item Receiver uses complementary \emph{private} key
          \begin{itemize}[nosep]
              \item Must be kept secret
          \end{itemize}
\end{itemize}

\begin{figure}[H]
    \tikzsetnextfilename{asymmetric-key}
    \begin{tikzpicture}
        \node (plaintext1) {Plaintext};
        \node[below=of plaintext1,draw,fill=yellow!40,ellipse,align=left] (enc) {Encrypt with\\\emph{public} key};
        \node[right=6cm of enc,draw,fill=yellow!40,ellipse,align=left] (dec) {Decrypt with\\\emph{private} key};
        \node[above=of dec] (plaintext2) {Plaintext};
        \path(enc) -- (dec) node[pos=0.5,minimum size=3cm,cloud,draw,fill=blue!40!white,yshift=-1cm] (internet) {};
        \node at ([yshift=1cm]internet.center) {Internet};
        \draw[->] (plaintext1) -- (enc);
        \draw (enc.south) -- ([yshift=-1cm]enc.south);
        \draw[->] ([yshift=-1cm]enc.south) -| (dec) node[pos=0.25,above] {Ciphertext};
        \draw[->] (dec) -- (plaintext2);
    \end{tikzpicture}
\end{figure}

\subsection{Works in Reverse Direction Too!}
\begin{itemize}[nosep]
    \item Sender uses own \emph{private} key
    \item Receiver uses complementary \emph{public} key
    \item Allows sender to prove he knows private key
\end{itemize}

\begin{figure}[H]
    \tikzsetnextfilename{asymmetric-key-rev}
    \begin{tikzpicture}
        \node (plaintext1) {Plaintext};
        \node[below=of plaintext1,draw,fill=yellow!40,ellipse,align=left] (enc) {Decrypt with\\\emph{public} key};
        \node[right=6cm of enc,draw,fill=yellow!40,ellipse,align=left] (dec) {Encrypt with\\\emph{private} key};
        \node[above=of dec] (plaintext2) {Plaintext};
        \path(enc) -- (dec) node[pos=0.5,minimum size=3cm,cloud,draw,fill=blue!40!white,yshift=-1cm] (internet) {};
        \node at ([yshift=1cm]internet.center) {Internet};
        \draw[->] (plaintext2) -- (dec);
        \draw (dec.south) -- ([yshift=-1cm]dec.south);
        \draw[->] ([yshift=-1cm]dec.south) -| (enc) node[pos=0.25,above] {Ciphertext};
        \draw[->] (enc) -- (plaintext1);
    \end{tikzpicture}
\end{figure}

\subsection{Integrity: Cryptographic Hashes}
\begin{itemize}[nosep]
    \item Sender computes a \emph{digest} of messages $\mathbf{m}$, i.e., $H(m)$
          \begin{itemize}[nosep]
              \item $H$ is a publicy known \emph{hash function}
          \end{itemize}
    \item Send $m$ in any manner
    \item Send digest $\mathbf{d} = H(\mathbf{m})$ to receiver in a secure way:
          \begin{itemize}[nosep]
              \item Using another physical channel
              \item Using encryption (why does this help?)
          \end{itemize}
    \item Upon receiving $\mathbf{m}$ and $\mathbf{d}$, receiver recomputs $H(\mathbf{m})$ to see whether result agrees with $\mathbf{d}$.
\end{itemize}

\subsection{Operation of Hashing for Integrity}
\begin{figure}[H]
    \tikzsetnextfilename{hasing-for-integrity}
    \begin{tikzpicture}
        \node (plaintext1) {Plaintext};
        \node[right=10cm of plaintext1] (plaintext2) {Plaintext};
        \node[ellipse,draw,fill=yellow!40,below right=3cm and 0.5 cm of plaintext1,align=center] (d1) {Digest\\(MD5)};
        \node[ellipse,draw,fill=yellow!40,below left=3cm and 0.5 cm of plaintext2,align=center] (d2) {Digest\\(MD5)};
        \node[above=1cm of d2,diamond,draw,fill=yellow!40,minimum width=2cm] (eq) {=};
        \draw[->] (plaintext1) -- ([yshift=-1cm]plaintext1.south) -| (d1);
        \path (d1) -- (d2) node[pos=0.5,cloud,fill=cyan!40!white,minimum height=4cm,minimum width=5cm,yshift=-1cm] (internet) {};
        \node at ([yshift=-0.5cm]internet.north) {Internet};
        \draw[->,red,ultra thick] (d1.south) -- ([yshift=-0.5cm]d1.south) -| node[pos=0.3,above] {\textcolor{black}{digest}} ([xshift=-0.5cm]eq.west) -- (eq.west);
        \node[above=0.8cm of eq,align=center] (c) {\emph{Corrupted}\\\emph{Message}};
        \draw[->] (plaintext1) -- ([yshift=-5cm]plaintext1.south) -| (plaintext2);
        \draw[->] (plaintext1) -- ([yshift=-5cm]plaintext1.south) -| (d2) -- (eq) node[pos=0.5,right] {$\text{digest}'$};
        \draw[->] (eq) -- (c) node[pos=0.4,right] {No};
    \end{tikzpicture}
\end{figure}

\subsection{Cryptographically Strong Hashes}
\begin{itemize}[nosep]
    \item Hard to find \emph{collisions}
          \begin{itemize}[nosep]
              \item Adversary can't find two inputs that produce same hash
              \item Someone cannot alter a message without modifying digest
              \item Can succinctly refer to large objects
          \end{itemize}
    \item Hard to \emph{invert}
          \begin{itemize}[nosep]
              \item Given hash, adversary can't find input that produces it
              \item Can refer obliquely to private objects (e.g. passwords)
                    \begin{itemize}[nosep]
                        \item Send hash of object rather than object itself
                    \end{itemize}
          \end{itemize}
\end{itemize}

\subsection{Effects of Cryptographic Hashing}
\begin{figure}[H]
    \tikzsetnextfilename{hash-values}
    \begin{tikzpicture}
        \node[draw,rectangle,fill=cyan!40,minimum width=2.5cm] (f1) {Fox};
        \node[draw,rectangle,fill=cyan!40,minimum width=2.5cm,align=center,below=2cm of f1] (f2) {The red fox\\ \emph{runs} across\\ the ice};
        \node[draw,rectangle,fill=cyan!40,minimum width=2.5cm,align=center,below=2cm of f2] (f3) {The red fox\\ \emph{walks} across\\ the ice};

        \foreach \x in {1,2,3}
            {
                \node[draw,rectangle,fill=yellow!40,right=of f\x] (h\x) {SHA-256};
                \draw[->,ultra thick] (f\x) -- (h\x);
            }
        \node[draw,rectangle,fill=pink!80,right=of h1,align=center] (sha1) {\texttt{6E9C620D CD316BF2}\\\texttt{9A37BF6D 0BE3D685}\\\texttt{ACFD18BE A7E6D5A2}\\\texttt{A6971045 39610491}};
        \draw[->,ultra thick] (h1) -- (sha1);
        \node[draw,rectangle,fill=pink!80,right=of h2,align=center] (sha2) {\texttt{57F1DD85 87DEE57A}\\\texttt{F424DD79 A679EE62}\\\texttt{FDA75462 4D2B1387}\\\texttt{E949F38E 36FD630D}};
        \draw[->,ultra thick] (h2) -- (sha2);
        \node[draw,rectangle,fill=pink!80,right=of h3,align=center] (sha3) {\texttt{E08EAB79 6F274B1E}\\\texttt{98E59150 F0B2CB22}\\\texttt{7B68F407 8172AE49}\\\texttt{D754752A 3DD30848}};
        \draw[->,ultra thick] (h3) -- (sha3);
    \end{tikzpicture}
\end{figure}

\subsection{Public Key Authentication}
\begin{itemize}[nosep]
    \item Each side only needs to know the other side's public key
          \begin{itemize}[nosep]
              \item No secret key needs to be shared
          \end{itemize}
    \item $A$ encrypts a \texttt{nonce} (random number) $x$ using $B$'s public key
    \item $B$ proves it can recover $x$
    \item $A$ can authenticate itself to $B$ in the same way
\end{itemize}

\begin{figure}[H]
    \tikzsetnextfilename{public-key-auth}
    \begin{tikzpicture}
        \node (a) {$A$};
        \node[right=4cm of a] (b) {$B$};
        \draw (a) -- ([yshift=-4cm]a.south);
        \draw (b) -- ([yshift=-4cm]b.south);
        \draw[->] ([yshift=-0.5cm]a.south) -- ([yshift=-1.5cm]b.south) node[above,pos=0.5,rotate=-10] {$E(x,\texttt{Public}_B)$};
        \draw[->] ([yshift=-1.5cm]b.south) -- ([yshift=-3cm]a.south) node[above,pos=0.5,rotate=15] {$x$};
    \end{tikzpicture}
\end{figure}

\subsection{Digital Signatures}
\begin{itemize}[nosep]
    \item Suppose Alice has published public key $K_E$
    \item If she wishes to prove who she is, she can send a message $x$ encrypted with her \emph{private} key $K_D$
          \begin{itemize}[nosep]
              \item Therefore: anyone with public key $K_E$ can recover $x$, verify that Alice must have sent the message
              \item It provides a \emph{digital signature}
              \item Alice can't later deny it $\implies$ \emph{non-repudiation}
          \end{itemize}
\end{itemize}

\begin{figure}[H]
    \tikzsetnextfilename{alice-bob-payment}
    \begin{tikzpicture}[every node/.style={minimum width=2cm,align=center}]
        \node[rectangle,draw,fill=cyan!40] (plaintext1) {I will\\pay \$500};
        \node[rectangle,draw,fill=yellow!40,right=of plaintext1,label=above:{Alice}] (encrypt) {Sign\\(Encrypt)};
        \node[rectangle,draw,fill=cyan!60!black,below=of encrypt] (crypt) {\texttt{DFCD3454}\\\texttt{BBEA788A}};
        \node[rectangle,draw,fill=yellow!40,below=of crypt,label=below:{Bob}] (decrypt) {Verify\\(Decrypt)};
        \node[rectangle,draw,fill=cyan!40,left=of decrypt] (plaintext2) {I will\\pay \$500};
        \draw[->,ultra thick] (plaintext1) -- (encrypt);
        \draw[->,ultra thick] (encrypt) -- (crypt);
        \draw[->,ultra thick] (crypt) -- (decrypt);
        \draw[->,ultra thick] (decrypt) -- (plaintext2);

        \node[right=of encrypt] (akey) {Alice's Private Key};
        \node[right=of decrypt] (bkey) {Alice's Public Key};
        \draw[->,ultra thick] (akey) -- (encrypt);
        \draw[->,ultra thick] (bkey) -- (decrypt);
    \end{tikzpicture}
\end{figure}

\subsection{Public Key Infrastructure (PKI)}
\begin{itemize}[nosep]
    \item Public key crypto is \emph{very} powerful\dots
    \item \dots by the \emph{realities} of tying public keys to real world identities turn out to be quite hard
    \item PKI: \emph{Trust distribution} mechanism
          \begin{itemize}[nosep]
              \item Authentication via \emph{Digital Certificates}
          \end{itemize}
    \item Trust doesn't mean someone is honest, just that they are who they say they are\dots
\end{itemize}

\subsection{Managing Trust}
\begin{itemize}[nosep]
    \item The most solid level of trust is rooted in our direct personal experience
          \begin{itemize}[nosep]
              \item e.g. Alice's trust that Bob is who they say they are
              \item Clearly doesn't scale to a global network!
          \end{itemize}
    \item In its absence, we rely on \emph{delegation}
          \begin{itemize}[nosep]
              \item Alice trusts Bob's identity because Charlie attests to it\dots
              \item \dots and Alice trusts Charlie
          \end{itemize}
    \item Trust is not particularly transitive
          \begin{itemize}[nosep]
              \item Should Alice trust Bob because she trusts Charlie\dots
              \item \dots and Charlie vouches for Donna \dots
              \item \dots and Donna says Eve is trustworthy\dots
              \item \dots and Eve vouces for Bob's identity?
          \end{itemize}
    \item Two models of delegating trust
          \begin{itemize}[nosep]
              \item Rely on your set of friends and their friends
                    \begin{itemize}[nosep]
                        \item ``Web of Trust'' -- e.g. PGP
                    \end{itemize}
              \item Rely on trust, well known authorities (\emph{and their minions})
                    \begin{itemize}[nosep]
                        \item ``Trusted root'' -- e.g. HTTPS
                    \end{itemize}
          \end{itemize}
\end{itemize}

\subsection{PKI Conceptual Framework}
\begin{itemize}[nosep]
    \item Trusted-Root PKI
          \begin{itemize}[nosep]
              \item Basis: well-known public key serves as \emph{root} of a hierarchy
              \item Managed by the Certificate Authority (CA)
          \end{itemize}
    \item To publish a public key, ask the CA to digitally sign a statement indicating that they agree (``certify'') that it is indeed your key
          \begin{itemize}[nosep]
              \item This is a \emph{certificate} for your key (certificate = bunch of bits)
                    \begin{itemize}[nosep]
                        \item Includes both your public key and the signed statement
                    \end{itemize}
              \item Anyone can verify the signature
          \end{itemize}
    \item Delegation of trust to the CA
          \begin{itemize}[nosep]
              \item They'd better not screw up (duped into signing bogus key)
              \item They'd better have procedures for dealing with stolen keys
              \item Note: can build up a \emph{hierarchy} of signing
          \end{itemize}
\end{itemize}

\subsection{HTTPS}
\begin{itemize}[nosep]
    \item Steps after clicking \url{https://www.amazon.com}
    \item https = ``use HTTP over SSL/TLS''
          \begin{itemize}[nosep]
              \item SSL = Secure Socket Layer
              \item TLS = Transport Layer Security
                    \begin{itemize}[nosep]
                        \item Successor to SSL, and compatible with it
                    \end{itemize}
              \item RFC 4346
          \end{itemize}
    \item Provides security layer (authentication, encryption) on top of TCP
          \begin{itemize}[nosep]
              \item Fairly transparent to the app
          \end{itemize}
\end{itemize}

\subsubsection{HTTPS Connection via SSL/TLS}
\begin{itemize}[nosep]
    \item Browser (client) connects via TCP to Amazon's \texttt{HTTPS} server
    \item Client sends over a list of crypto protocols it supports
    \item Server picks protocols to use for this session
    \item Server sends over its cerificate
    \item (all of this is in the clear)
\end{itemize}

\subsection{Inside the Server's Certificate}
\begin{itemize}[nosep]
    \item \emph{Name} associated with the cert (e.g. Amazon)
    \item Amazon's \emph{public key}
    \item A bunch of auxhiliary info (physical address, type of cert, expiration time)
    \item URL to \emph{revocation center} to check for revoked keys
    \item Name of certificate's \emph{signatory} (who signed it)
    \item A public-key signature of a has (\emph{MD5}) of all this
          \begin{itemize}[nosep]
              \item Constructed using the signatory's private RSA key
          \end{itemize}
\end{itemize}

\subsection{Validating Amazon's Identity}
\begin{itemize}[nosep]
    \item Example: certificate of entity Amazon
          \[\texttt{Cert} = E\left(\set{\text{Amazon}, \text{KAmazon}_{\text{public}}}, \text{KCA}_{\text{private}}\right)\]
    \item Browser retrieves cert belonging to the \emph{signatory}
          \begin{itemize}[nosep]
              \item These are \emph{hardwired into the browser}
          \end{itemize}
    \item If it can't find the cert, it warns the user that the site has not been verified
          \begin{itemize}[nosep]
              \item And may ask whether to continue
              \item Note, can still proceed, just \emph{without authentication}
          \end{itemize}
    \item Browser uses public key in signatory's cert to decrypt signature
    \item Assuming signature matches, not have high confidence it is indeed Amazon
          \begin{itemize}[nosep]
              \item \dots assuming signatory is trustworth
          \end{itemize}
\end{itemize}

\subsection{HTTPS Connection (SSL/TLS)}
\begin{itemize}[nosep]
    \item Browser constructs a random \emph{session key} $K$
    \item Browser encrypts $K$ using Amazon's public key
    \item Browser sends $E(K, KA_{\text{public}})$ to server
    \item Browser displays lock
    \item All subsequent communication encrypted with symmetric cipher using key $K$
          \begin{itemize}[nosep]
              \item e.g. client can authentication using a password
          \end{itemize}
\end{itemize}

\subsection{Pharming}
\begin{itemize}[nosep]
    \item How can we get web clients to redirect to malicious sites?
    \item Name resolution
          \begin{itemize}[nosep]
              \item Send a query to a DNS
              \item Trust the IP address returned by the DNS
              \item Other ways to go from name to IP?
          \end{itemize}
\end{itemize}