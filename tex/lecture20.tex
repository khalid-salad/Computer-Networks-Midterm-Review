\section{Swiching and Physical Layer}
\subsection{Spanning Tree}
\begin{tcolorbox}[colframe=red!40!white,colback=yellow!20,title=Spanning tree]
    In the mathematical field of graph theory, a spanning tree $T$ of an undirected graph $G$ is a subgraph that is a tree which includes all of the vertices of $G$. In general, a graph may have several spanning trees, but a graph that is not connected will not contain a spanning tree (see spanning forests below). If all of the edges of $G$ are also edges of a spanning tree $T$ of $G$, then $G$ is a tree and is identical to $T$ (that is, a tree has a unique spanning tree and it is itself).
\end{tcolorbox}
\url{https://en.wikipedia.org/wiki/Spanning_tree}

\subsection{Spanning Tree Algorithms}
\begin{itemize}[nosep]
    \item Graph search algorithms
    \item Dijkstra's algorithm
    \item Minimum-spanning Tree Algorithms
\end{itemize}

\subsection{Spanning Tree}
\begin{itemize}[nosep]
    \item Need to disable ports, so that no loops in the network
    \item Like creating a spanning tree in a graph
          \begin{itemize}[nosep]
              \item View switches and networks as nodes, ports as edges
          \end{itemize}
\end{itemize}

\subsection{Distributed Spanning Tree Algorithm}
\begin{itemize}[nosep]
    \item Every bridge has a unique ID (Ethernet Address)
    \item Goal:
          \begin{itemize}[nosep]
              \item Bridge with the smallest ID is the root
              \item Each segment has one designated bridge, responsible for forwarding its packets towards the root
                    \begin{itemize}[nosep]
                        \item Bridge closest to root is designated bridge
                        \item If there is a tie, bridge with lowest ID wins
                    \end{itemize}
          \end{itemize}
\end{itemize}

\subsection{Spanning Tree Protocol}
\begin{itemize}[nosep]
    \item Spanning Tree messages contain:
          \begin{itemize}[nosep]
              \item ID of bridge sending the message
              \item ID sender believes to be the root
              \item Distance (in hops) from sender to root
          \end{itemize}
    \item Bridges remember best config msg on each port
    \item Send message when you think you are the root
    \item Otherwise, forward messages from best known root
          \begin{itemize}[nosep]
              \item Add one to distance before forwarding
              \item Don't forward if you know you aren't dedicated bridge
          \end{itemize}
\end{itemize}

\subsection{Limitations of Bridges}
\begin{itemize}[nosep]
    \item Scaling
          \begin{itemize}[nosep]
              \item Spanning tree algorithm doesn't scale
              \item Broadcast does not scale
              \item No way to route around congested links, \emph{even if path exists}
          \end{itemize}
    \item May violate assumptions
          \begin{itemize}[nosep]
              \item Could confuse some applications that assume single segment
              \item Much more likely to drop packets
              \item Makes latency between nodes non-uniform
              \item Beware of transparency
          \end{itemize}
\end{itemize}

\subsection{Local Area Network}
\subsection{Physical Layer}
\begin{itemize}[nosep]
    \item Responsible for specifying the physical medium
          \begin{itemize}[nosep]
              \item Type of cable, fiber, wireless frequency
          \end{itemize}
    \item Responsible for specifying the signal (modulation)
          \begin{itemize}[nosep]
              \item Transmitter varies \emph{something} (amplitude, frequency, phase)
              \item Receiver samples, recovers signal
          \end{itemize}
    \item Responsible for specifying the bits (encoding)
          \begin{itemize}[nosep]
              \item Bits above physical layer $\to$ \emph{chips}
          \end{itemize}
\end{itemize}
\subsection{Specifying the signal}
\begin{itemize}[nosep]
    \item Chips vs. bits
          \begin{itemize}[nosep]
              \item Chips: data (in bits) at the physical layer
              \item Bits: data above the physical layer
          \end{itemize}
    \item Physical layer specifies Analog signal $\Leftrightarrow$ chip mapping
          \begin{itemize}[nosep]
              \item On-ioff keying (OOK): voltage of 0 is 0, $+V$ is 1
              \item PAM-5: 000 is 0, 001 is $+1$, 010 is $-1$, 011 is $-2$, 100 is $+2$
              \item Frequency shift keying (FSK)
              \item Phase shift keying (PSK)
          \end{itemize}
\end{itemize}

\subsection{Modulation}
\begin{itemize}[nosep]
    \item Specifies mapping between digital signal and some variation in analog signal
    \item Why not just a square wave ($1V=1$, $0V=0$)?
          \begin{itemize}[nosep]
              \item Not square when bandwidth limited
          \end{itemize}
    \item Bandwidth -- frequencies that a channel propagates well
          \begin{itemize}[nosep]
              \item Signals consist of many frequency components
              \item Attenuation and delay frequency-dependent
          \end{itemize}
\end{itemize}

\subsection{Use Carriers}
\begin{itemize}[nosep]
    \item Idea: use only frequencies that transmit well
    \item \emph{Modulate} the signal to encode bits
\end{itemize}

\subsection{How Fast can you Really Send?}
\begin{itemize}[nosep]
    \item Depends on frequency and signal/noise ratio
    \item Shannong $C=B\log_2(1+\sfrac{S}{N})$
          \begin{itemize}[nosep]
              \item $C$ is the channel capacity in bits/second
              \item $B$ is the bandwidth of the channel in \si{Hz}
              \item $S$ and $N$ are average signal and noise power
          \end{itemize}
    \item Example: Telephone Line
          \begin{itemize}[nosep]
              \item $3\si{KHz}$ bandwith, $30\si{dB}\sfrac{S}{N} = 10^{\sfrac{30}{10}} = 1000$
              \item $C\approx 30\si{Kbps}$
          \end{itemize}
\end{itemize}

\subsection{Encoding}
\begin{itemize}[nosep]
    \item Now assume that we can somehow modulate a signal: receiver can decode our binary stream
    \item How od we encode binary data onto signals?
    \item One approach: Non-return to Zero (NRZ)
          \begin{itemize}[nosep]
              \item Transmit 0 as low, 1 as high!
          \end{itemize}
\end{itemize}

\subsection{Drawbacks of NRZ}
\begin{itemize}[nosep]
    \item No signal could be interpreted as 0 (or vice-versa)
    \item Consecutive 1s or 0s are problematic
    \item Baseline wander problem
          \begin{itemize}[nosep]
              \item How do you set the threshold?
              \item Could compare to average, but average may drift
          \end{itemize}
    \item Clock recovery problem
          \begin{itemize}[nosep]
              \item For long runs of no change, could miscount periods
          \end{itemize}
\end{itemize}

\subsection{Alternative Encodings}
\begin{itemize}[nosep]
    \item Non-return to Zero Inverted (NRZI)
          \begin{itemize}[nosep]
              \item Encode 1 with transition from current signal
              \item Encode 0 by staying at the same level
              \item At least solves problem of consecutive 1s
          \end{itemize}
\end{itemize}

\subsection{Manchester}
\begin{itemize}[nosep]
    \item Map $0\to\text{chips }01$; $1\to\text{chips }10$
          \begin{itemize}[nosep]
              \item Transmission rate now 1 bit per two clock cycles
          \end{itemize}
    \item Solves clock recovery, baseline wander
    \item But cuts transmission rate in half
\end{itemize}

\subsection{4B/5B}
\begin{itemize}[nosep]
    \item Can we have a more efficient encoding?
    \item Every 4 bits encoded as 4 \emph{chips}
    \item Need 16 5-bit codes:
          \begin{itemize}[nosep]
              \item selected to have no more than one leading 0 and no more than two trailing 0s
              \item Never get more than 3 consecutive 0s
          \end{itemize}
    \item Transmit chips using NRZI
    \item Other codes used for other purposes
          \begin{itemize}[nosep]
              \item e.g. 11111: line idle; 00100: half
          \end{itemize}
    \item Achieves 80\% efficiency
\end{itemize}

\subsection{Encoding Goals}
\begin{itemize}[nosep]
    \item DC Balancing (same number of 0 and 1 chips)
    \item Clock synchronization
    \item Can recover some chip errors
    \item Constrain analog signal patterns to make signal more robust
    \item Want near channel capacity with negligible errors
          \begin{itemize}[nosep]
              \item Shannon says it's possible, doesn't tell us how
              \item Codes can get computationally expensive
          \end{itemize}
    \item In practice
          \begin{itemize}[nosep]
              \item More complex encoding: fewer bps, more robust
              \item Less complex encoding: more bps, less robust
          \end{itemize}
\end{itemize}

\subsection{802.15.4}
\begin{itemize}[nosep]
    \item Standard for low-power, low-rate wireless PANs
          \begin{itemize}[nosep]
              \item Must tolerate high chip error rates
          \end{itemize}
    \item Uses a 4B/32B bit-to-chip encoding
\end{itemize}

\begin{figure}[H]
    \tikzsetnextfilename{802.15.4encoding}
    \begin{tikzpicture}[every node/.style={rectangle},node distance=0cm]
        \node (0) {0000};
        \node[right=1cm of 0] (c0) {11011001110000110101001000101110};
        \node[below=of 0] (1) {0001};
        \node[below=of c0] (c1) {11101101100111000011010100100010};
        \node[below=of 1] (2) {0010};
        \node[below=of c1] (c2) {00101110110110011100001101010010};
        \node[below=of 2] (3) {0011};
        \node[below=of c2] (c3) {00100010111011011001110000110101};
        \node[below=of 3] (4) {0100};
        \node[below=of c3] (c4) {01010010001011101101100111000011};
        \node[below=of 4] (5) {0101};
        \node[below=of c4] (c5) {00110101001000101110110110011100};
        \node[below=of 5] (6) {0110};
        \node[below=of c5] (c6) {11000011010100100010111011011001};
        \node[below=of 6] (7) {0111};
        \node[below=of c6] (c7) {10011100001101010010001011101101};
        \node[below=of 7] (8) {1000};
        \node[below=of c7] (c8) {10001100100101100000011101111011};
        \node[below=of 8] (9) {1001};
        \node[below=of c8] (c9) {10111000110010010110000001110111};
        \node[below=of 9] (10) {1010};
        \node[below=of c9] (c10) {01111011100011001001011000000111};
        \node[below=of 10] (11) {1011};
        \node[below=of c10] (c11) {01110111101110001100100101100000};
        \node[below=of 11] (12) {1100};
        \node[below=of c11] (c12) {00000111011110111000110010010110};
        \node[below=of 12] (13) {1101};
        \node[below=of c12] (c13) {01100000011101111011100011001001};
        \node[below=of 13] (14) {1110};
        \node[below=of c13] (c14) {10010110000001110111101110001100};
        \node[below=of 14] (15) {1111};
        \node[below=of c14] (c15) {11001001011000000111011110111000};
        \foreach \i in {0,...,15} {
                \draw[->] (\i) -- (c\i);
                \draw (\i.north west) -- (\i.north east);
                \draw (c\i.north west) -- (c\i.north east);
            }
        \draw (15.south west) -- (15.south east);
        \draw (c15.south west) -- (c15.south east);
        \draw (0.north east) -- (15.south east);
        \draw (0.north west) -- (15.south west);
        \draw (c0.north east) -- (c15.south east);
        \draw (c0.north west) -- (c15.south west);

        \draw [decorate,decoration={brace,amplitude=10pt,mirror,raise=4pt},yshift=0pt] (c0.north east) -- (c0.north west) node[above,black,midway,yshift=0.4cm] {Chips};
        \draw [decorate,decoration={brace,amplitude=10pt,mirror,raise=4pt},yshift=0pt] (0.north east) -- (0.north west) node[above,black,midway,yshift=0.4cm] {Bits};
        \draw [decorate,decoration={brace,amplitude=10pt,mirror,raise=4pt},yshift=0pt] (0.north west) -- (15.south west) node[left,black,midway,xshift=-0.7cm,yshift=-0.7cm,rotate=-90] {Symbols};
    \end{tikzpicture}
\end{figure}

\subsection{Framing}
\begin{itemize}[nosep]
    \item Given a stream of bits, how can we represent boundaries?
    \item Break sequence of bits into a frame
    \item Typically done by a network adaptor
\end{itemize}

\subsection{Representing Boundaries}
\begin{itemize}[nosep]
    \item Sentinels
    \item Length counts
    \item Clock-based
\end{itemize}

\subsection{Bit-Oriented Protocols}
\begin{itemize}[nosep]
    \item View message as a stream of bits, not bytes
    \item Can use sentinel approach as well, (e.g. HDLC)
          \begin{itemize}[nosep]
              \item HDLC begin/end sequence 01111110
          \end{itemize}
    \item Use \emph{bit stuffing} to escape 01111110
          \begin{itemize}[nosep]
              \item Always append 0 after five consecutive 1s in data
              \item After five 1s, receiver uses next two bits to decide if stuffed, end of frame, or error
          \end{itemize}
\end{itemize}

\subsection{Length-based Framing}
\begin{itemize}[nosep]
    \item Drawback of sentinel techniques
          \begin{itemize}[nosep]
              \item Length of frame depends on data
          \end{itemize}
    \item Alternative: put length in header (e.g. DDCMP)
    \item Dancer: Framing Errors
          \begin{itemize}[nosep]
              \item What if high bit counter gets corrupted?
              \item Adds 8K to length of frame, may lose many frames
              \item CRC checksum helps detect error
          \end{itemize}
\end{itemize}