\section{TCP Friendliness and Getting Help from the Network}
\subsection{TCP Friendliness}
\begin{itemize}[nosep]
    \item Can other protocols co-exist with TCP?
          \begin{itemize}[nosep]
              \item e.g. if you want to write a video streaming app using UDP, how to do congestion control?
          \end{itemize}
    \item Equation-based Congestion Control
          \begin{itemize}[nosep]
              \item Instead of implementing TCP's CC, estimate the rate at which TCP would send. Function of what?
              \item RTT, MSS, Loss
          \end{itemize}
    \item Measure RTT, Loss, send at that rate!
\end{itemize}
\subsection{TCP Throughput}
\begin{itemize}[nosep]
    \item Assume a TCP connection of window $W$, rount-trip time of \texttt{RTT}, segment size of \texttt{MSS}
          \begin{itemize}[nosep]
              \item Sending Rate $S = W \times \sfrac{\texttt{MSS}}{\texttt{RTT}}$ \textcolor{red}{(1)}
          \end{itemize}
    \item Drop $W = \sfrac{W}{2}$
          \begin{itemize}[nosep]
              \item grows by $\texttt{MSS} \sfrac{W}{2}$ RTTs, until another drop at $W\approx W$
          \end{itemize}
    \item Average window then $0.75\times S$
          \begin{itemize}[nosep]
              \item From (1), $S = 0.75 W \sfrac{\texttt{MSS}}{\texttt{RTT}}$ \textcolor{red}{(2)}
          \end{itemize}
    \item Loss rate is 1 in number of packets between losses:
          \begin{align*}\texttt{Loss} &= \frac{1}{1+\left(\sfrac{W}{2} + \sfrac{W}{2} + 1 + \sfrac{W}{2} + 2 + \dots + W\right)}\\&=\frac{1}{\sfrac{3}{8}W^2}\textcolor{red}{(3)}\end{align*}
    \item $\texttt{Loss} = \sfrac{8}{(3W^2)} \implies W = \sqrt{\frac{8}{3\cdot\texttt{Loss}}}$ \textcolor{red}{(4)}
    \item Substituting (4) in (2), $S = 0.75 W \sfrac{\texttt{MSS}}{\texttt{RTT}}$
          \[\texttt{Throughput} \approx 1.22\times\frac{\texttt{MSS}}{\texttt{RTT}\cdot\sqrt{\texttt{Loss}}}\]
    \item Equation-based rate control can be TCP friendly and have better properties, e.g., small jitter, fast ramp-up\dots
\end{itemize}

\[W = \sqrt{\frac{8}{3p}}\]
Substitute $W$ into the bandwidth equation below:
\[\texttt{BW} = \frac{\text{data per cycle}}{\text{time per cycle}} = \frac{\texttt{MSS}\cdot\frac{3}{8}W^2}{\texttt{RTT}\cdot\frac{W}{2}} = \frac{\sfrac{\texttt{MSS}}{p}}{\texttt{RTT}\sqrt{\frac{2}{3p}}}\]
Collect the constants in one term, $C=\sqrt{\sfrac{3}{2}}$, then we arrive at
\[\texttt{BW} = \frac{\texttt{MSS}}{\texttt{RTT}}\frac{C}{\sqrt{p}}\]

\subsection{What happens when Link is Lossy}
\begin{itemize}[nosep]
    \item $\text{Throughput} = \sfrac{1}{\sqrt{\text{Loss}}}$
\end{itemize}
\subsection{What can we do about it?}
\begin{itemize}[nosep]
    \item Two types of losses: congestion and corrupt
    \item One option: mask corruption losses from TCP
          \begin{itemize}[nosep]
              \item Retranmissions at the link layer
              \item e.g. snoop TCP: intercept duplicate acknowledgments, retransmit locally, filter them from the sender
          \end{itemize}
    \item Another option:
          \begin{itemize}[nosep]
              \item Tell the sender about the cause for the drop
              \item Requires modification of the TCP endpoints
          \end{itemize}
\end{itemize}
\subsection{Congestion Avoidance}
\begin{itemize}[nosep]
    \item TCP creates congestion to then back off
          \begin{itemize}[nosep]
              \item Queues at bottleneck link are often full: increased delay
              \item Sawtooth pattern: jitter
          \end{itemize}
    \item Alternative strategy
          \begin{itemize}[nosep]
              \item Predict when congestion is about to happen
              \item Reduce rate early
          \end{itemize}
    \item Two approaches
          \begin{itemize}[nosep]
              \item Host centric: TCP vegas
              \item Router-centric: RED, DECBit
          \end{itemize}
\end{itemize}
\subsection{TCP Vegas}
\begin{itemize}[nosep]
    \item Idea: source watches for sign that router's queue is building up (e.g. sending rate flattens)
    \item Compare Actual Rate ($A$) with Expected Rate ($E$)
          \begin{itemize}[nosep]
              \item If $E - A > \beta$, decrease \texttt{cwnd} linearly: $A$ isn't responding
              \item If $E - A < \alpha$, increase \texttt{cwnd} lienarly: room for $A$ to grow
          \end{itemize}
\end{itemize}
\subsection{Vegas}
\begin{itemize}[nosep]
    \item Shorter router queues
    \item Lower jitter
    \item Problem:
          \begin{itemize}[nosep]
              \item Doesn't compete well with Reno. Why?
              \item Reacts earlier, Reno is more aggressive, ends up with higher bandwidth\dots
          \end{itemize}
\end{itemize}
\subsection{Help from the network}
\begin{itemize}[nosep]
    \item What if routers could \emph{tell} TCP that congestion is happening?
          \begin{itemize}[nosep]
              \item Congestion causes queues to grow: rate mismatch
          \end{itemize}
    \item TCP responds to drops
          \begin{itemize}[nosep]
              \item Idea: Random Early Drop (RED)
                    \begin{itemize}[nosep]
                        \item Rather than wait for queue to become full, drop packet with some probability that increases with queue length
                        \item TCP will react by reducing \texttt{cwnd}
                        \item Could also mark instead of dropping: ECN
                    \end{itemize}
          \end{itemize}
\end{itemize}

\subsection{RED Details}
\begin{itemize}[nosep]
    \item Computer average queue length (EWMA)
          \begin{itemize}[nosep]
              \item Don't want to react to very quick fluctuations
          \end{itemize}
\end{itemize}
\subsection{RED Drop Probability}
\begin{itemize}[nosep]
    \item Define two thresholsd: \texttt{MinThresh}, \texttt{MaxThresh}
    \item Drop probability:
\end{itemize}
\begin{figure}[H]
    \tikzsetnextfilename{red-drop-prob}
    \begin{tikzpicture}
        \begin{axis}[
                xmin=0,xmax=6.25,
                ymin=0,ymax=1.5,
                axis lines = left,
                xtick={1,3},
                ytick={0.50,1},
                xticklabels={\texttt{MinThresh}, \texttt{MaxThresh}},
                yticklabels={\texttt{MaxP}, 1.0},
                ylabel = $P$ (drop),
                xlabel = \texttt{AvgLen},
                x label style={at={(axis description cs:1.0,0.1)},anchor=north},
                y label style={at={(axis description cs:0.1,1.1)},rotate=-90,anchor=north},
            ]
            \addplot[color=blue,domain=0:1,thick]{0};
            \addplot[color=blue,domain=1:3,thick]{0.25*x-0.25};
            \addplot[color=blue,domain=3:6,thick]{1};

            \draw[blue,thick] (3,0.5) -- (3,1);
        \end{axis}
    \end{tikzpicture}
\end{figure}
\begin{itemize}[nosep]
    \item Improvements to spread drops (see book)
\end{itemize}
\subsection{RED Avantages}
\begin{itemize}[nosep]
    \item Probability of dropping a packet of a particular flow is roughly proportional to the share of the bandwidth that flow is currently getting
    \item Higher network utilization with low delays
    \item Average queue length small, but can absorb bursts
    \item ECN
          \begin{itemize}[nosep]
              \item Similar to RED, but router sends bit in the packet
              \item Must be supported by both ends
              \item Avoids retransmissions optionally dropped packets
          \end{itemize}
\end{itemize}
\subsection{More help from the network}
\begin{itemize}[nosep]
    \item Problem: still vulnerable to malicious flows!
          \begin{itemize}[nosep]
              \item RED will drop packets from large flows preferentially, but they don't have to respond appropriately
          \end{itemize}
    \item Idea: Multiple Queues (one per flow)
          \begin{itemize}[nosep]
              \item Serve queues in Round-Robin
              \item Nagle (1987)
              \item Good: protects against misbehaving flows
              \item Disadvantage?
              \item \emph{Flows with larger packets get higher bandwidth}
          \end{itemize}
\end{itemize}
