\section{Distance Vector, Link State, and Inter-AS Routing}
\subsection{Routing}
\subsection{Good news travels fast}
\begin{figure}[H]
    \tikzsetnextfilename{goodnews}
    \begin{tikzpicture}[every node/.style={draw=blue,circle},node distance=2.5cm]
        \node (a) {$A$};
        \node[above right=1.5 and 2 of a] (b) {$B$};
        \node[below right=1.5 and 2 of b] (c) {$C$};
        \draw[thick] (a) -- (b) node[pos=0.5,draw=none,above] (s) {4};
        \draw[thick] (a) -- (c) node[pos=0.5,draw=none,below] {10};
        \draw[thick] (b) -- (c) node[pos=0.5,draw=none,above] {1};

        \node[red,thick,draw,circle] at (s) (w) {};
        \draw[red,thick] (w.south east) -- (w.north west) node[above left,red,draw=none] {1};
    \end{tikzpicture}
\end{figure}
\begin{itemize}[nosep]
    \item A decrease in link cost has to be fresh information
    \item Network converges in $\bigO{d}$ steps ($d$ is the diameter)
\end{itemize}

\subsection{Bad news travels slowly}
\begin{figure}[H]
    \tikzsetnextfilename{badnews}
    \begin{tikzpicture}[every node/.style={draw=blue,circle},node distance=2.5cm]
        \node (a) {$A$};
        \node[above right=1.5 and 2 of a] (b) {$B$};
        \node[below right=1.5 and 2 of b] (c) {$C$};
        \draw[thick] (a) -- (b) node[pos=0.5,draw=none,above] (s) {4};
        \draw[thick] (a) -- (c) node[pos=0.5,draw=none,below] {10};
        \draw[thick] (b) -- (c) node[pos=0.5,draw=none,above] {1};

        \node[red,thick,draw,circle] at (s) (w) {};
        \draw[red,thick] (w.south east) -- (w.north west) node[above left,red,draw=none] {12};
    \end{tikzpicture}
\end{figure}
\begin{itemize}[nosep]
    \item An increase in cost may cause confusion with old information
    \item May form loops
\end{itemize}
\subsection{How to avoid loops}
\begin{itemize}[nosep]
    \item IP TTL field prevents a packet from living forever
          \begin{itemize}[nosep]
              \item Does \emph{not} repair a loop
          \end{itemize}
    \item Simple approach: consider a small cost $n$ (e.g. 16) to be infinity
          \begin{itemize}[nosep]
              \item After $n$ rounds decide node is unavailable
              \item But rounds can be long, this takes time
          \end{itemize}
\end{itemize}
\subsection{Better loop avoidance}
\begin{itemize}[nosep]
    \item Split Horizon
          \begin{itemize}[nosep]
              \item When sending updates to node $A$, don't include routes you learned from $A$
              \item Prevents $B$ and $C$ from sending cost 2 to $A$
          \end{itemize}
    \item Split Horizon with Poison Reverse
          \begin{itemize}[nosep]
              \item Rather than advertising routes learned from $A$, explicitly include cost of $\infty$
              \item Faster to break out of loops, but increases advertisement sizes
          \end{itemize}
\end{itemize}
\subsection{Warning}
\begin{itemize}[nosep]
    \item Split Horizon/Split Horizon with Poison Reverse only helps between nodes
          \begin{itemize}[nosep]
              \item Can still get a loop with three nodes involved
              \item Might need to delay advertising routes after changes, but affects convergence time
          \end{itemize}
\end{itemize}

\subsection{Link State Routing}
\begin{itemize}[nosep]
    \item Strategy
          \begin{itemize}[nosep]
              \item send to all nodes information about directly connected neighbors
          \end{itemize}
    \item Link State Packet (LSP)
          \begin{itemize}[nosep]
              \item ID of the node that created the LSP
              \item Cost of link to each directly connected neighbor
              \item Sequence number (\texttt{SEQNO})
              \item TTL
          \end{itemize}
\end{itemize}

\subsection{Reliable Flooding}
\begin{itemize}[nosep]
    \item Store most recent LSP from each node
          \begin{itemize}[nosep]
              \item Ignore earlier versions of the same LSP
          \end{itemize}
    \item Forward LSP to all nodes but the one that sent it
    \item Generate new LSP periodically
          \begin{itemize}[nosep]
              \item Increment \texttt{SEQNO}
          \end{itemize}
    \item Start at $\texttt{SEQNO}=0$ when reboot
          \begin{itemize}[nosep]
              \item If you hear your own packet with $\texttt{SEQNO} = n$, set your next \texttt{SEQNO} to $n + 1$
          \end{itemize}
    \item Decrement TTL of each stored LSP
          \begin{itemize}[nosep]
              \item Discard when $\texttt{TTL} = 0$
          \end{itemize}
\end{itemize}
\subsection{Calculating best path}
\nameref*{alg:dijkstra} computes the shortest path from node $s$ (``yourself'') to every other node in the graph. Let $\texttt{Nodes}$ denote the set of nodes in the graph, $\Call{Weight}{i, j}$ the weight of the edge between $i$ and $j$ ($\infty$ if there is no edge), $\Call{Cost}{n}$ the cost of the path from $s$ to $n$, and $\Call{Route}{n}$ the next node to visit in the path from $s$ to $n$.
\begin{algorithm}[H]
    \caption[Dijkstra's Algorithm]{\nameref*{alg:dijkstra}}
    \label{alg:dijkstra}
    \begin{algorithmic}[1]
        \State $\texttt{unvisited} \gets \texttt{Nodes} - \set{s}$
        \ForEach{$n\in \texttt{unvisited}$}
        \State $\Call{Cost}{n} \gets \Call{Weight}{s, n}$
        \If{$\Call{Weight}{s, n} < \infty$}
        \State $\Call{Route}{n} \gets n$
        \Else
        \State $\Call{Route}{n} \gets \Null$
        \EndIf
        \EndForEach
        \While{there are nodes in \texttt{unvisited}}
        \State let $w$ be the node in \texttt{unvisited} with lowest value $\Call{Cost}{w}$
        \State $\texttt{unvisited}.\Call{remove}{w}$
        \ForEach{node $n\in\texttt{unvisited}$}
        \If{$\Call{Cost}{w} + \Call{Weight}{w, n} < \Call{Cost}{n}$}
        \State $\Call{Cost}{n} \gets \Call{Cost}{w} + \Call{Weight}{w, n}$
        \State $\Call{Route}{n} \gets \Call{Route}{w}$
        \EndIf
        \EndForEach
        \EndWhile
    \end{algorithmic}
\end{algorithm}

\subsection{Distance Vector vs. Link State}
\begin{itemize}[nosep]
    \item Number of messages (per node $v$ with degree $d$)
          \begin{itemize}[nosep]
              \item DV: $\bigO{d}$
              \item LS: $\bigO{nd}$ for $n$ nodes in the system
          \end{itemize}
    \item Computation
          \begin{itemize}[nosep]
              \item DV: convergence time varies (e.g. count-to-infinity)
              \item LS: $\bigO{n^2}$ with $\bigO{nd}$ messages
          \end{itemize}
    \item Robustness: what happens with malfunctioning router?
          \begin{itemize}[nosep]
              \item DV:
                    \begin{itemize}[nosep]
                        \item Nodes can advertise incorrect \emph{path} cost
                        \item Others can use the cost, propagates through the network
                    \end{itemize}
              \item LS:
                    \begin{itemize}[nosep]
                        \item Nodes can advertise incorrect \emph{link} cost
                    \end{itemize}
          \end{itemize}
\end{itemize}
\subsection{Examples}
\begin{itemize}[nosep]
    \item RIPv2
          \begin{itemize}[nosep]
              \item Fairly simple implementation of DV
              \item RFC 2453 (38 pages)
          \end{itemize}
    \item OSPF (Open Shortest Path First)
          \begin{itemize}[nosep]
              \item More complex link-state protocol
              \item Adds notion of \emph{areas} for scalability
              \item RFC 2328 (244 pages)
          \end{itemize}
\end{itemize}
\subsection{RIPv2}
\begin{itemize}[nosep]
    \item Runs on UDP port 520
    \item Link cost = 1
    \item Periodic updates every 30 seconds, plus triggered updates
    \item Relies on count-to-infinity to resolve loops
          \begin{itemize}[nosep]
              \item Maximum diameter 15 ($\infty = 16$)
              \item Supports Split Horizon, Poison Reverse
          \end{itemize}
\end{itemize}
\subsection{Packet Format}
\begin{verbatim}
0                   1                   2                   3
0 1 2 3 4 5 6 7 8 9 0 1 2 3 4 5 6 7 8 9 0 1 2 3 4 5 6 7 8 9 0 1
+-+-+-+-+-+-+-+-+-+-+-+-+-+-+-+-+-+-+-+-+-+-+-+-+-+-+-+-+-+-+-+-+
| command (1) | version (1) |           must be zero (2)        |
+---------------+---------------+-------------------------------+
|                                                               |
~                         RIP Entry (20)                        ~
|                                                               |
+---------------+---------------+---------------+---------------+
\end{verbatim}
\subsection{RIPv2 Entry}
\begin{verbatim}
0                   1                   2                   3
0 1 2 3 4 5 6 7 8 9 0 1 2 3 4 5 6 7 8 9 0 1 2 3 4 5 6 7 8 9 0 1
+-+-+-+-+-+-+-+-+-+-+-+-+-+-+-+-+-+-+-+-+-+-+-+-+-+-+-+-+-+-+-+-+
| address family identifier (2) |      Route Tag (2)            |
+-------------------------------+-------------------------------+
|                        IP address (4)                         |
+---------------------------------------------------------------+
|                        Subnet Mask (4)                        |
+---------------------------------------------------------------+
|                        Next Hop (4)                           |
+---------------------------------------------------------------+
|                        Metric (4)                             |
+---------------------------------------------------------------+
\end{verbatim}
\subsection{Next Hop Field}
\begin{itemize}[nosep]
    \item Allows one router to advertise routes for multiple routers on the same subnet
    \item Suppose only \texttt{XR1} talks RIPv2
\end{itemize}
\begin{verbatim}
-----   -----   -----           -----   -----   -----
|IR1|   |IR2|   |IR3|           |XR1|   |XR2|   |XR3|
--+--   --+--   --+--           --+--   --+--   --+--
  |       |       |               |       |       |
--+-------+-------+---------------+-------+-------+--
  <-------------RIP-2------------->
\end{verbatim}
\subsection{OSPFv2}
\begin{itemize}[nosep]
    \item Link state protocol
    \item Runs directly over IP (protocol 89)
          \begin{itemize}[nosep]
              \item Has to provide its own reliability
          \end{itemize}
    \item All exchanges are authenticated
    \item Adds notion of \emph{areas} for scalability
\end{itemize}
\subsection{Inter-Domain Routing}
\subsection{Why Inter vs. Intra}
\begin{itemize}[nosep]
    \item Why not just use OSPF everywhere?
          \begin{itemize}[nosep]
              \item e.g. hierarchies of OSPF areas?
              \item Hint: scaling is not the only limitation
          \end{itemize}
    \item BGP is a policy control and information hiding protocol
          \begin{itemize}[nosep]
              \item intra = trusted, inter = untrusted
              \item Different policies by different ASs
              \item Diffrent costs by different ASs
          \end{itemize}
\end{itemize}